\par	This chapter will provide an introduction to the basics of formulating the equations of Multi-Body Dynamics
	(MBD) for numerical solving. The equations are based on the field of classic physics, i.e. Newtons famous equations
	of force equillibrium, but with emphasis on formulating the equations to fit solving by nummerical methods.
\psfig{htb}{.4}{pendulum1}{The pendulum. A system frequently chosen to illustrate the concept of multi-body dynamics. 
	  The pendulum consists of a ground part and a pendulum attached to the ground part by a revolute joint.}{fig:pendulum1}	
	
\section{Introduction}	
\par Multi-Body Dynamics describes the combined kinetics and kinematics of systems consisting of one or more bodies.
	These bodies may be connected in various ways, and furthermore exerted to various external applied forces and 
	torques. Multi-Body Dynamics describes planar (2D) as well as spatial (3D) system, and while simple planar 
	systems can be handled without the use of advanced tools like Msc.Adams, the complexity of 3D systems 
	causes most people to loose track of the set of equations.
\par In traditional MBD all bodies (or links) are considered rigid, but with mass and dimension properties. A 
	beam has a mass, a mass moment of inertia and physical dimensions - but the flexibility is usually not 
	taken into account. A basic system could be a pendulum as shown on \fref{fig:pendulum1}.
	
\section{Basic equations of motion}
\par Deriving the set of equations to describe motion of one or more connected bodies requires a few steps of
	breaking down the system into manageble pieces. The process is briefly described here and the pendulum is finally
	applied as an example.

\subsection{Free-body diagram (applied forces)}
\par The free-body diagram (fbd) is well known to most people. It is the art of drawing a simplified outline of the
	structure and applied forces. This is known from the world of statics, where the purpose is to ballance the
	sum of equations in each direction -- when balance is achieved, the system is in equilibrium. For a 2D problem
	with axes named $x,y,\theta$, the static equations related to the free-body diagram are:

\begin{subequations} \label{eqn:static1}
	\begin{align}
	&\sum{F_x} = 0 \\
	&\sum{F_y} = 0 \\
	&\sum{T_{\theta}}  = 0
	\end{align}
\end{subequations}

\par By using vector notation with cartesian coordinates $[x,y ] $, equation \eref{eqn:static1} can be rewritten to;

\begin{subequations}
 	\begin{align}
 	 &\sum{\ve{F}[x,y]} = 0 \\
 	 &\sum{\ve{T}_{\theta}} = 0 
 	\end{align}
\end{subequations}


\subsection{Kinematic diagram (path of motion)}
\par A mechanical structure consisting of one or more bodies interconnected by various joints will have a limited
	degree of motion. Acceleration of each point on each body can be described by kinematic equations. Later on 
	we will use kinematic equations to relate inertial properties of the bodies to the applied forces to evaluate
	acceleration of each body.
\par As the kinematic equations describe the possible path of motion of the system, the equations are very much 
	depending the restriction imposed on the system. However, the kinematic equations will always describe the 
	interconnection between acceleration in the different axis. In the 2D problem mentioned above, the kinematic
	equations will desribe the connection between linear acceleration in $x$ and $y$ direction $a_x$ and $a_y$ 
 	respectively and the rotary
	acceleration $\alpha$. Throughout these notes, the time-wise derivative will be denoted by a dot hence the accelerations
	are denoted:
	
\begin{subequations}
 	\begin{align}
 	 \frac{d^2}{dt^2}x &= \ddot{x}  = a_x\\
 	 \frac{d^2}{dt^2}y &= \ddot{y}  = a_y\\
 	 \frac{d^2}{dt^2}\theta &= \ddot{\theta} = \alpha
 	\end{align}
\end{subequations}

\par As before, it can be convenient to use vector notation for accelerations as well. In order for this to apply
	we need to define the vector $\ve{r}$ as a vector between two points with cartesian coordinates $x$ and $y$.

\begin{equation}
 	 \npdev{\ve{r}}{t}{2} = \ddot{\ve{r}} = \ve{a}
\end{equation}


\subsection{Kinetic diagram (motion caused by forces)}
\par The kinetic diagram and the kinetic equations is the output of combining the applied forces with the kinematic
	equation of motion and the inertial properties of the bodies. In this set of equations the sum of forces are e

\subsection{Example: Pendulum}

\section{Concept of reference frames}
\subsection{Descriptive vectors}

\subsection{Basic entities in MBD}
\par There exist three typical terms that are frequently reffered when discussing MBD; bodies, constraints and applied
	forces.

\subsubsection{Body}
\par The term body refers to the physical element of any given size. Several attributes are related to a body. 
	First of all a body has a mass, thus it has a mass moment of inertia and a center of gravity. It would
	typically also have some special points of interest. These points may be location of links with other bodies
	or the point of action for any external applied force or torque.
\subsubsection{Constraint}
\par Constraints limits the degrees of freedom for a mechanical system. It provides a way to formulate various
	kinds of joints, contacts etc. A constraint will typically formulate a given relation between two points
	on different bodies.
	
\psfig{htb}{.55}{pendulum2}{The definitions are here used to illustrate the different parts of the pendulum shown 
	  on \fref{fig:pendulum1}.}{fig:pendulum2}	
	  	
\subsubsection{Applied forces}
\par Bodies and constraints formulate a kinematic system i.e. a mechanical system described as a mechanism. If
	one of the bodies are set in motion, the constraints will determine the motion of the remaining system. 
	Applied foreces and torques are the input to the systemj. The applied forces and torques will accelerate
	the bodies, the acceleration will change the configuation of the system and the constraints will transfer
	reaction forces in joints from body to body and the system moves as a kinetic system within the kinematic
	boundaries.
	
\section{Formulating equation for numerical solving}

\subsection{Rotation matrix}

\section{Rotation sequence method}

\section{Further reading}
\par Several authors have contributed with different ways of formulating equations of multi-body dynamics for 
	numerical computation. Amongst these authors are: \cite{nikravesh} and \cite{shabana}