\par	This chapter will provide an introduction to the basics of formulating the equations of Multi-Body Dynamics
	(MBD) for numerical solving. The equations are based on the field of classic physics, i.e. Newtons famous equations
	of force equillibrium, but with emphasis on formulating the equations to fit solving by nummerical methods.

	
\section{Introduction}	
\par Multi-Body Dynamics describes the combined kinetics and kinematics of systems consisting of one or more bodies.
	These bodies may be connected in various ways, and furthermore exerted to various external applied forces and 
	torques. Multi-Body Dynamics describes planar (2D) as well as spatial (3D) system, and while simple planar 
	systems can be handled without the use of advanced tools like Msc.Adams, the complexity of 3D systems 
	causes most people to loose track of the set of equations.
\par In traditional MBD all bodies (or links) are considered rigid, but with mass and dimension properties. A 
	beam has a mass, a mass moment of inertia and physical dimensions - but the flexibility is usually not 
	taken into account. 
	
\subsection{Basic entities in MBD}
\par There exist three typical terms that are frequently reffered when discussing MBD; bodies, constraints and applied
	forces.

\subsubsection{Body}
\par The term body refers to the physical element of any given size. Several attributes are related to a body. 
	First of all a body has a mass, thus it has a mass moment of inertia and a center of gravity. It would
	typically also have some special points of interest. These points may be location of links with other bodies
	or the point of action for any external applied force or torque.
\subsubsection{Constraint}
\par Constraints limits the degrees of freedom for a mechanical system. It provides a way to formulate various
	kinds of joints, contacts etc. A constraint will typically formulate a given relation between two points
	on different bodies.
\subsubsection{Applied forces}
\par Bodies and constraints formulate a kinematic system i.e. a mechanical system described as a mechanism. If
	one of the bodies are set in motion, the constraints will determine the motion of the remaining system. 
	Applied foreces and torques are the input to the systemj. The applied forces and torques will accelerate
	the bodies, the acceleration will change the configuation of the system and the constraints will transfer
	reaction forces in joints from body to body and the system moves as a kinetic system within the kinematic
	boundaries.
	
	



\section{Basic equations of motion}

\section{Concept of reference frames}
\subsection{Descriptive vectors}

\section{Formulating equation for numerical solving}

\subsection{Rotation matrix}

\section{Rotation sequence method}